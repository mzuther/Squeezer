\documentclass[paper=a5,pagesize=auto,twoside=false,fontsize=12pt,DIV=classic,BCOR=0mm,headinclude=true,footinclude=false]{scrbook}

%% font selection
\usepackage{utopia}
\usepackage{avant}

%% redefine headers
\usepackage[headsepline]{scrpage2}
\pagestyle{scrheadings}

\setkomafont{author}{%
  \normalfont\normalsize
}

\setkomafont{date}{%
  \normalfont\normalsize
}

\setkomafont{dedication}{%
  \raggedright\normalfont\small\itshape
}

\setkomafont{pageheadfoot}{%
  \normalfont\itshape
}

\automark[chapter]{chapter}
\renewcommand{\chaptermark}[1]{\markboth{#1}{#1}}
%\renewcommand{\sectionmark}[1]{\markright{#1}}

%% additional KOMA script options
\KOMAoptions{parskip=half,headsepline=true,numbers=noendperiod,captions=tableheading}
\recalctypearea{} % recalculate type area after selecting fonts

\usepackage{array}
\usepackage[british]{babel}
\usepackage{bbding}
\usepackage{fancyvrb}
\usepackage{float}  % direct positioning of floats
\usepackage{graphicx}
\usepackage{ifpdf}
\usepackage[utf8x]{inputenc}
\usepackage{paralist}
\usepackage{ucs}
\usepackage[detect-all]{siunitx}
\usepackage[hyphens,obeyspaces]{url}
\usepackage{varioref}
\usepackage{wrapfig}  % must be included after package "float"
\usepackage[dvipsnames]{xcolor}

%% Make sure "hyperref" comes last of your loaded packages, to give it
%% a fighting chance of not being over-written, since its job is to
%% redefine many LATEX commands.
\ifpdf
	\usepackage[unicode,pdftex]{hyperref}
\else
	\usepackage[unicode,hypertex]{hyperref}
\fi

\hypersetup{
  unicode=true,                % non-Latin characters in Acrobat’s bookmarks
  pdftitle=Squeezer,           % title
  pdfauthor=Martin Zuther,     % author
  pdfsubject=Squeezer,         % subject of the document
  pdfcreator=Martin Zuther,    % creator of the document
  colorlinks=true,             % false: boxed links; true: colored links
  linkcolor=blue,              % color of internal links
  citecolor=blue,              % color of links to bibliography
  filecolor=blue,              % color of file links
  urlcolor=blue                % color of external links
}

%% KOMA script captions
\addtokomafont{caption}{\small}
\addtokomafont{captionlabel}{\bfseries}
\setcapindent{0em}

%% taken from package "l2tabu"
\tolerance 1414
\hbadness 1414
\emergencystretch 1.5em
\hfuzz 0.3pt
\widowpenalty=10000
\vfuzz
\hfuzz
\raggedbottom

%% package "fancyvrb"
\DefineVerbatimEnvironment{VerbatimBoth}{Verbatim}{gobble=2,samepage=true,frame=single,framerule=0.4mm,framesep=2mm,xleftmargin=4mm,xrightmargin=4mm,rulecolor=\color{BurntOrange},label=32 and 64 bit}

\DefineVerbatimEnvironment{Verbatim32}{Verbatim}{gobble=2,samepage=true,frame=single,framerule=0.4mm,framesep=2mm,xleftmargin=4mm,xrightmargin=4mm,rulecolor=\color{OrangeRed},label=32 bit}

\DefineVerbatimEnvironment{Verbatim64}{Verbatim}{gobble=2,samepage=true,frame=single,framerule=0.4mm,framesep=2mm,xleftmargin=4mm,xrightmargin=4mm,rulecolor=\color{Cerulean},label=64 bit}

%% package "varioref"
\labelformat{chapter}{chapter~#1}
\labelformat{section}{section~#1}
\labelformat{subsection}{section~#1}
\labelformat{subsubsection}{section~#1}

\labelformat{figure}{figure~#1}
\labelformat{table}{table~#1}

%% package "paralist"
\setdefaultitem{\textbullet}{$\circ$}{}{}

%% package "url"
\urlstyle{rm}

%% package "wrapfig"
\setlength{\intextsep}{0.2\baselineskip}

%% package "siunitx"
\sisetup{per-mode=symbol,locale=UK}

\DeclareSIUnit[]\bits{bits}
\DeclareSIUnit[]\dB{dB}
\DeclareSIUnit[]\dBFS{dB\,FS}
\DeclareSIUnit[]\dBr{dBr}
\DeclareSIUnit[]\dBSPL{dB\,SPL}
\DeclareSIUnit[]\dBu{dBu}
\DeclareSIUnit[]\LK{LK}
\DeclareSIUnit[]\LKFS{LKFS}
\DeclareSIUnit[]\VRMS{V\textsubscript{RMS}}
\DeclareSIUnit[]\VU{VU}

%% scaling of screenshots
\newcommand{\screenshotscale}{0.7}

%% layout
\usepackage[inner=18mm,outer=18mm,top=26mm,bottom=27mm,headsep=6.5mm,headheight=5mm]{geometry}

% %% start new page for every section
% \let\stdsection\section  
% \renewcommand\section{\newpage\stdsection}

%% command "application"
\DeclareRobustCommand*{\application}[1]{\texttt{#1}}


%%% Local Variables:
%%% mode: latex
%%% mode: outline-minor
%%% TeX-command-default: "Rubber"
%%% TeX-master: "../squeezer"
%%% TeX-PDF-mode: t
%%% ispell-local-dictionary: "british"
%%% End:

\input{include/hyphenation.sty}

\title{Squeezer}
\author{Martin Zuther}

\begin{document}

\title{Squeezer}

\subtitle{
  \normalsize{\textrm{\textmd{
        \vfill
        Flexible general-purpose audio compressor \\
        with a touch of citrus
        \vfill
        \vspace{1.5em}
        \includegraphics[scale=0.30,clip]{include/images/squeezer.png}
        \vfill
      }}}
}

\author{}

\date{\emph{Last edited on \today}}

\dedication{
  \includegraphics[scale=0.65,clip]{include/images/cc-by-sa.png}
  \vspace{0.25em}

  This documentation by \href{http://www.mzuther.de/}{Martin Zuther}
  is licensed under a
  \href{http://creativecommons.org/licenses/by-sa/4.0/}{Creative
    Commons Attribution-ShareAlike 4.0 International License} with the
  exception of trademark logos.

  \vspace{2.5em}

  \includegraphics[scale=0.55,clip]{include/images/VST_Compatible_Logo_Steinberg_negative.png}

  VST is a trademark of Steinberg Media Technologies GmbH,
  registered in Europe and other countries.
}

\maketitle

\tableofcontents

\clearpage  % layout

\chapter{About Squeezer}
\label{chap:about_squeezer}

\emph{I wrote this compressor to learn all I could about audio
  compression.  It took me several months to gather all the specs and
  papers I needed and convert this vast amount of information into
  code.}

What makes Squeezer different from the compressors I know is the
stepping of all knobs to \emph{preset values} (it took a loooong time
to fine-tune them).  On clicking a small red light located right next
to them, however, the knobs become \emph{continuous}.  That allows you
to find good-sounding settings very fast while not preventing
fine-tuning.

Squeezer is also very flexible.  Its internal layout can be changed in
many ways: feed-forward / feed-back, linear / S-curve / log, optical /
transistor, RMS / almost peak, three different knees.

By writing and fine-tuning this compressor I have gained deep insight
into audio compressors and really learned how to use them musically.
But not only that, Squeezer has quickly become my go-to generic
compressor.

\chapter{What is compression?}
\label{chap:what_is_compression}

\section{The tale of the bar keeper}
\label{sec:compression_tale_of_bar_keeper}

Once upon a time there was a bar.  Customers enjoyed their drinks and
music played in the background.  After a while, the bar keeper noticed
that his drinks sold best when the music had a certain loudness.
Unfortunately, almost every record they played had a different
loudness.

So he bought a loudness meter and wrote a set of instructions for the
DJ:

\begin{tabular}{p{0.2 \textwidth}p{0.8 \textwidth}}

  \textbf{\SI{80}{\dBSPL}} &
  set mixer's output gain to \SI{0}{\dB} \\[0.5em]

  \textbf{\SI{85}{\dBSPL}} &
  set mixer's output gain to \SI{-5}{\dB} \\[0.5em]

  \textbf{\SI{90}{\dBSPL}} &
  set mixer's output gain to \SI{-10}{\dB} \\[0.5em]

  \textbf{\SI{95}{\dBSPL}} &
  set mixer's output gain to \SI{-15}{\dB} \\[0.25em]

\end{tabular}

This worked quite well and he could finally afford to buy a shiny new
bar counter.  Upon which the DJ gave notice (he had always dreamt of
getting a \emph{stereo} mixer, so the new counter didn't go down too
well with him).  After an initial shock, the bar keeper was quite
content.  The DJ really \emph{had} been lousy.  But how to keep the
high number of drinks?

Being an entrepreneur, the bar keeper took the challenge and connected
the mixer's output to a simple amplifier.  The amplifier's input gain
was regulated by a level sensor which in turn was connected to the
amplifier's output -- a classic negative feed-back loop.

Although the new device exactly followed the DJ's instructions (and
didn't smell as bad), customers complained about its bad sound.  The
bar keeper had invented a \emph{wave shaping} device -- gain changes
were applied instantly which \emph{distorted} its output signal.

Fortunately, the bar keeper remembered a very important thing: the DJ
had often been drunk and taken his time to apply gain changes.  So the
bar keeper improved his device by smoothing the level sensor's output.

He could have become very rich.  But he didn't realise the magnitude
of his invention and continued being a happy bar keeper.  Which is not
the worst thing, when you come to think of it.

To this day, however, sound engineers fall silent in awe when they
hear his name -- the name of the bar keeper who \textbf{invented the
  compressor}.

\chapter{Installation}
\label{chap:installation}

In order to use the pre-compiled binaries, simply extract the Squeezer
files from the downloaded archive.  For the plug-ins, you'll then have
to move the extracted files to your respective plug-in folder.

Squeezer requires a processor which supports the SSE2 instruction set.
On Windows, you might also have to install the
\href{https://www.visualstudio.com/downloads/}{Visual C++
  Redistributable for Visual Studio 2017}.

Should the stand-alone version ever fail to start, you can reset its
settings by deleting the file \path{squeezer_stereo.ini} or
\path{squeezer_mono.ini}.  These files are located in \path{~/.config}
(GNU/Linux) or \path{%appdata%\.config\} (Windows).

\chapter{Final words}
\label{chap:final_words}

I want to express my gratitude to the \textbf{Audio Engineering
  Society} for their wonderful E-Library.  This includes everybody who
wrote the fine papers within the library, especially Dimitrios
Giannoulis, Michael Massberg and Joshua D. Reiss.

I must also thank the \textbf{beta testers} and \textbf{users of
  Squeezer} for sending kind words, suggestions and bug reports.
Finally, I want to thank the \textbf{open source community} for making
all of this possible.

Although coding Squeezer has been a lot of fun, it has also been a lot
of work.  So if you like Squeezer, why not send me a short email and
tell me so?  Write a few words about yourself, send suggestions for
future updates or volunteer to create a nice skin.  I also really
enjoy listening to music that you have produced using my software
\dots

Here is my email address (please remove ``\texttt{-nospam}''):

\begin{center}
  \texttt{"Martin Zuther" <code-nospam@mzuther.de>}
\end{center}

Thanks for using free software.  I hope you'll enjoy it!

\appendix

\chapter{Build Squeezer}
\label{chap:build_squeezer}

\section{Dependencies}
\label{sec:dependencies}

\subsection{premake}
\label{sec:dependencies_premake}

\begin{tabbing}
  \hspace*{6em}\=\=\kill

  Importance:  \> required \\
  Version:     \> 5.0.0 (alpha13) \\
  License:     \> BSD \\
  Homepage:    \> \href{https://premake.github.io/}{premake.github.io}
\end{tabbing}

\subsubsection{Installation}

Place the binary somewhere in your \path{PATH}.  Depending on your
platform, you should run \path{premake} using the scripts
\path{Builds/run_premake.sh} or \path{Builds/run_premake.bat}.

To change the premake file using the provided Jinja templates, you'll
also have to install the necessary dependencies.

\subsection{JUCE library}

\begin{tabbing}
  \hspace*{6em}\=\=\kill

  Importance:  \> required \\
  Version:     \> 5.3.2 \\
  License:     \> ISC and GPL v3 (among others) \\
  Homepage:    \> \href{http://www.juce.com/}{www.juce.com}
\end{tabbing}

\subsubsection{Installation}

Extract the archive into the directory \path{libraries/juce}.

If you want to build the LV2 plug-in, please extract the archive
\path{distrho_lv2-xxxxxxxx.tar.gz} into the same directory.

\subsection{Virtual Studio Technology SDK}

\begin{tabbing}
  \hspace*{6em}\=\=\kill

  Importance:  \> optional \\
  Version:     \> 2.4 / 3.6.8 \\
  License:     \> proprietary / GPL v3 \\
  Homepage:    \> \href{http://www.steinberg.net/en/company/developer.html}{www.steinberg.net}
\end{tabbing}

\subsubsection{Installation}

Just extract the archive into the directory \path{libraries/vst}.

\subsection{Python}

\begin{tabbing}
  \hspace*{6em}\=\=\kill

  Importance:  \> optional \\
  Version:     \> 3.5 (or higher) \\
  License:     \> Python Software Foundation License \\
  Homepage:    \> \href{http://www.python.org/}{www.python.org}
\end{tabbing}

You'll only need Python if you want to change the premake file (see
\ref{sec:dependencies_premake}) using Jinja templates.

\subsubsection{Installation (Windows)}

You can download an installer from the website.

\subsection{Jinja}

\begin{tabbing}
  \hspace*{6em}\=\=\kill

  Importance:  \> optional \\
  Version:     \> 2.8 (or higher) \\
  License:     \> BSD \\
  Homepage:    \> \href{http://jinja.pocoo.org/}{jinja.pocoo.org}
\end{tabbing}

You'll only need Jinja if you want to change the premake file using
templates (see \ref{sec:dependencies_premake}).

\subsection{Artistic Style}

\begin{tabbing}
  \hspace*{6em}\=\=\kill

  Importance:  \> optional \\
  Version:     \> 2.05.1 \\
  License:     \> LGPL v3 \\
  Homepage:    \> \href{http://astyle.sourceforge.net/}{astyle.sourceforge.net}
\end{tabbing}

This application formats the code so it looks more beautiful and
consistent.  Thus, you only have to install it if you plan to help me
with coding Squeezer.

\subsubsection{Installation}

Place the binary somewhere in your \path{PATH}.  Depending on your
platform, you should run \path{astyle} using the scripts
\path{Source/format_code.sh} or \path{Source/format_code.bat}.

\newpage %% layout

\section{GNU/Linux}

\subsection{Environment}

To build Squeezer yourself, I recommend setting up a \texttt{chroot}
environment.  This is fast and easy to do on Debian-based systems and
might save you a \textbf{lot} of trouble.  At the time of writing, I'm
using Linux Mint 19, but the procedure should be similar on your
distribution of choice.

Start by installing the necessary packages:

\begin{VerbatimBoth}
  sudo apt-get install debootstrap schroot
\end{VerbatimBoth}

Then install the \texttt{chroot} base system by executing the
following statements:

\begin{Verbatim32}
  sudo debootstrap --variant=buildd \
    --arch i386 bionic \
    /srv/chroot/bionic_i386 \
    http://archive.ubuntu.com/ubuntu
\end{Verbatim32}

\begin{Verbatim64}
  sudo debootstrap --variant=buildd \
    --arch amd64 bionic \
    /srv/chroot/bionic_amd64 \
    http://archive.ubuntu.com/ubuntu
\end{Verbatim64}

Running \path{debootstrap} will take some time.  Meanwhile, add the
following lines to \path{/etc/schroot/schroot.conf} (make sure you
remove all preceding white space so that each line begins in the first
column):

\begin{VerbatimBoth}
  [bionic-i386]
  description=Ubuntu bionic (i386)
  directory=/srv/chroot/bionic_i386
  profile=default
  personality=linux32
  type=directory
  users=username

  [bionic-amd64]
  description=Ubuntu bionic (amd64)
  directory=/srv/chroot/bionic_amd64
  profile=default
  personality=linux
  type=directory
  users=username
\end{VerbatimBoth}

Please make the necessary changes to \texttt{username}.  If you
experience problems, you can try to change \texttt{bionic} to a
release name such as \texttt{wheezy}.

When \path{debootstrap} is done, log in as superuser:

\begin{Verbatim32}
  sudo schroot -c bionic-i386
\end{Verbatim32}

\begin{Verbatim64}
  sudo schroot -c bionic-amd64
\end{Verbatim64}

You'll have to change the file \path{/etc/apt/sources.list} first
(ignore the line break, it should be a single line):

\begin{VerbatimBoth}
  deb http://archive.ubuntu.com/ubuntu bionic
  main restricted universe
\end{VerbatimBoth}

Now install a few packages -- \path{less} and \path{vim} are optional,
but might come in handy:

\begin{VerbatimBoth}
  apt-get update
  apt-get -y install bash-completion clang \
    libasound2-dev libjack-jackd2-dev \
    mesa-common-dev xorg-dev less vim
  apt-get clean
\end{VerbatimBoth}

If you like \path{bash} completion, you might also want to open the
file \path{/etc/bash.bashrc} and unquote these lines:

\begin{VerbatimBoth}
  # enable bash completion in interactive shells
  if [...]
    [a couple of lines...]
  fi
\end{VerbatimBoth}

Finally, log out and log in as normal user:

\begin{Verbatim32}
  schroot -c bionic-i386
\end{Verbatim32}

\begin{Verbatim64}
  schroot -c bionic-amd64
\end{Verbatim64}

In this \path{chroot} shell, install the dependencies
(\ref{sec:dependencies}).  Congratulations -- you are now ready to
build Squeezer!

\subsection{Build}

After preparing the dependencies, start your \texttt{chroot}
environment

\begin{Verbatim32}
  schroot -c bionic-i386
\end{Verbatim32}

\begin{Verbatim64}
  schroot -c bionic-amd64
\end{Verbatim64}

change into the directory \path{build} and execute

\begin{VerbatimBoth}
  ./run_premake.sh
  make config=CFG TARGET
\end{VerbatimBoth}

where \application{CFG} is one of \application{debug\_x32},
\application{debug\_x64}, \application{release\_x32} and
\application{release\_x64}, and \application{TARGET} is the version
you want to compile, such as \application{linux\_standalone\_stereo}.

In case you run into problems, you can try to switch compilers by
opening the file \texttt{run\_premake.sh} and using the premake
options \texttt{--cc=clang} or \texttt{--cc=gcc}.

The compiled binaries will end up in the directory \path{bin}.

\section{Microsoft Windows}

\subsection{Build}

After preparing the dependencies, change into the directory
\path{build} and execute

\begin{VerbatimBoth}
  ./run_premake.bat
\end{VerbatimBoth}

Then change into the directory \path{Builds/windows/vs20xx}, open the
project file with the corresponding version of Visual C++ and build
the project.

The compiled binaries will end up in the directory \path{bin}.

\chapter{Licenses}

\scriptsize
\input{include/gpl_v3.tex}
\normalsize

\scriptsize
\input{include/cc-by-sa-4.0.tex}
\normalsize

\end{document}


%%% Local Variables:
%%% mode: latex
%%% mode: outline-minor
%%% TeX-command-default: "Rubber"
%%% TeX-master: t
%%% TeX-PDF-mode: t
%%% ispell-local-dictionary: "british"
%%% End:
